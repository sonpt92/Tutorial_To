\documentclass[12pt]{article}
\usepackage[utf8]{inputenc}
\usepackage{amsmath, amsthm, latexsym, amssymb, graphicx, bold-extra, mathrsfs, frcursive}
\DeclareMathOperator{\sh}{sh}
\DeclareMathOperator{\ch}{ch}
\DeclareMathOperator{\s}{s}
%\DeclareMathOperator{\c}{c}

\usepackage[pdftex]{color}
\usepackage[T1]{fontenc}
\usepackage{bm}

% VN
\usepackage[vietnamese]{babel}

%Insert image
\usepackage{graphicx}

% Simplifies margin settings
\usepackage{geometry}
\geometry{margin=20mm}

% Puts list item indicators in bold; makes flush with previous margin
\renewcommand\labelenumi{\bf\theenumi.}
\renewcommand\labelenumii{\bf\theenumii.}
% setlength\leftmargini{1.4em}
\setlength\leftmarginii{1.4em}

% Flexibility for headers and footers
\usepackage{fancyhdr}
\pagestyle{fancyplain}
\fancyhf{} %clear all header and footer fields
%\lhead{\bf \small --- \hspace*{\fill} Page \thepage}
\headsep 0.2in
\thispagestyle{empty}
\renewcommand{\headrulewidth}{0pt}
\renewcommand{\footrulewidth}{0pt}

\parindent 3ex % Paragraph identity, spacing on the first line of a paragraph
\parskip 10pt
\setlength{\headheight}{20pt}

% inkscape
\usepackage{color}
\usepackage{transparent}
\usepackage{epsfig}

% Table
\usepackage{array}
\newcolumntype{P}[1]{>{\centering\arraybackslash}p{#1}}
\title{ĐẠO HÀM}
\date{Ngày 2 tháng 3 năm 2020}
\author{Buổi 2}

\begin{document}
%	\maketitle
	\begin{center}
		\textbf{Toán buổi 2: ĐẠO HÀM}
	\end{center}
	\section{Một số quy tắc tính đạo hàm}
	Cho 2 hàm số của $x$: $u = u(x), \; v = v(x)$. Đạo hàm tương ứng là: $u' = \frac{du}{dx},\; v' = \frac{dv}{dx}$
	\begin{itemize}
		\item $(u\pm v)' = u' \pm v' $
		\item $(uv)' = u'v + v'u$
		\item $(\frac{u}{v})' = \frac{u'v - v'u}{v^2} $
		\item $[u(v)]' = u'_v . v'_x$
	\end{itemize}

	\section{Áp dụng}
	\subsection{Chứng minh}
	Cho 3 hàm số của $x$: $u = u(x), \; v = v(x), \; w = w(x)$. Chứng minh:
	$$(uvw)' = u'vw + uv'w + uvw' $$
	
	\subsection{Bài tập }
	Tìm tập xác định (Ensemble de définition) và tính đạo hàm của các hàm số sau:
	\begin{enumerate}
		\item $x^5-2\cos x$
		\item $\frac{x+2}{x^2-3x} $
		\item $\sin x + \cos x $
		\item $\sin \left(\frac{2x^2-2}{x+5}\right)$
		\item $xe^{-x} $ (BAC S 2017)
		\item $\frac{1}{2} (e^x + e^{-x} - 2) $ (BAC S 2018)
		\item $\sqrt{x} $
		\item $ \sqrt{x} + \frac{1}{\sqrt{x}}$
		\item $\ln(\sqrt{x}) $
		\item $\tan(2x) $
	\end{enumerate}
\end{document}